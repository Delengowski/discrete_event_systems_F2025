\documentclass[lettersize,journal]{IEEEtran}
\usepackage{biblatex}
\usepackage{amsmath,amsfonts}
\usepackage{algorithmic}
\usepackage{algorithm}
\usepackage{array}
\usepackage[caption=false,font=normalsize,labelfont=sf,textfont=sf]{subfig}
\usepackage{textcomp}
\usepackage{stfloats}
\usepackage{url}
\usepackage{verbatim}
\usepackage{graphicx}
\usepackage{hyperref}
\hyphenation{op-tical net-works semi-conduc-tor IEEE-Xplore}
% updated with editorial comments 8/9/2021
\addbibresource{main_paper.bib}
\begin{document}

\title{Extended Resource Conflict Checking and Resolution Controller
Design for Cross-Organization Emergency Response Processes}

\author{Matt Delengowski}

% The paper headers
\markboth{Rowan University Discrete Event Systems (ECE09568) Fall 2025}%
{Shell \MakeLowercase{\textit{et al.}}: A Sample Article Using
IEEEtran.cls for IEEE Journals}

\maketitle

\begin{abstract}
  Discrete Event Systems (DES) models of dynamical systems which
  change in discrete points of time rather than continuously in time,
  and often asynchronously. Examples of DES are in manufacturing,
  logistics, healthcare, and service operations.
  This paper part of a final project requirement for a DES course.
  The requirement for the final project is to pick a subject from DES
  and a related IEEE paper, which was written in the last 5 years,
  and is from a journal that has impact factor $> 5.0$.
  The topic of this paper is attempting to extend the work of
  \cite{main} which uses Petri Nets (PN) to model emergency response
  systems to optimize the allocation and dispersion of resources from
  first responders and down stream to hospitals.
  The extension is recreate the work of \cite{main} and then attempt
  to make it even more efficient.
\end{abstract}

\begin{IEEEkeywords}
  Cross-organization emergency response
  processes, performance evaluation, Petri nets, resolution
  controller design, resource conflict checking.
\end{IEEEkeywords}

\section{Introduction}
\IEEEPARstart{T}{his} .

\section{The Design, Intent, and \\ Limitations of the Templates}

\section{Text}

\section{Some Common Elements}
\subsection{Sections and Subsections}
\subsection{Lists}

\section{Conclusion}
The conclusion goes here.

{\appendix[Source Code]
  The source code for conducting experiments and this paper can be
  found at a
  \href{https://github.com/Delengowski/discrete_event_systems_F2025}{public
  github repository.}
  \url{https://github.com/Delengowski/discrete_event_systems_F2025}
}

\printbibliography

\end{document}
